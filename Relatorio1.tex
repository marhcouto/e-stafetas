\documentclass[12pt]{article}


\usepackage{geometry}
\usepackage{ulem}
\usepackage{import}



\date{16 de Abril de 2021}

\title{\Huge{\textbf{E-Stafetas - transportes de mercadorias em veículos elétricos} \\
\LARGE{ Conceção e Análise de Algoritmos - Parte I }}}

% Author
\author{
\Large{Class 3, group 2} \vspace{0.5em} \\
\begin{tabular}{r l}
	\email{up201906086@edu.fe.up.pt} & Marcelo Henriques Couto        \\
	\email{up201807481@edu.fe.up.pt} & Afonso Marques Morais Cabral de Carvalho \\
	\email{up201705110@edu.fe.up.pt} & João Luis Cardoso Rodrigo \\
\end{tabular}
}

\begin{document}

\maketitle
\newpage

{\Huge Indíce} \\
\newpage

\section{Descrição do Problema}

\subsection{Introdução}
Uma empresa de entrega de mercadorias ao domicílio pretende implementar um sistema capaz de gerir as rotas dos seus veículos. O tema deste projeto deriva o seu nome dos veículos elétricos que a empresa pretende utilizar nas entregas. Por este motivo, é preciso considerar diversos fatores que tornam esta uma situação diferente do usual, principalmente relacionados com a autonomia dos veículos.

\subsection{Entrega}
Os veículos terão que diariamente realizar múltiplas entregas. Cada entrega resume-se a dois passos:
\begin{itemize}
	\item Recolha do(s) produto(s) na loja ou centro de recolha
	\item Entrega do(s) produto(s) na morada correspondente
\end{itemize}

\subsection{Problema da Autonomia}
Para além disto, é necessário ter em conta a autonomia do veículo. Este fator restringe as rotas que poderão ser tomadas entre os pontos de interesse referidos anteriormente. Como solução são nos propostas duas estratégias diferentes. A primeira resume-se a que, sempre que um veículo não tenha autonomia suficiente, o caminho para a entrega deverá passar pela sede da empresa, de modo que este seja recarregado aí. A outra estratégia implica o cálculo de um caminho que passe por um dos múltiplos pontos de recarga que poderão estar espalhados pelo mapa. De certa forma, a primeira estratégia resume-se a uma situação específica da segunda, em que o ponto de recarga é único e coincide com a garagem da empresa.

\subsection{Problema da Imprevisibilidade}
Um outro problema com o qual é necessário lidar é o da imprevisibilidade do estado de um dado caminho. A qualquer altura, obras na via, acidentes e outros eventos semelhantes podem tornar um certo caminho impossível, levando à necessidade de recalcular os caminhos mais curtos ou de rotular a entrega impossível nas condições desse dado momento.

\subsection{Otimização das Viagens}
Como o número de veículos que a empresa tem disponíveis para realizar as entregas é reduzido, é necessário otimizar o seu uso.

\newpage

\section{Formalização do Problema}

% Input
\subsection{Input Data}
G = (V, A) -- grafo pesado dirigido representativo do mapa, composto por Vértices e Arestas
\begin{itemize}
	\item V – conjunto dos vértices (representam \uline{pontos de interesse} ou simplesmente pontos da rede viária) (V(i) é o i-ésimo elemento)
\begin{itemize}
	\item Adj $ \subseteq $  A – arestas que partem do vértice
	\item lat – latitude do ponto no mapa real
	\item long – longitude do ponto no mapa real

\end{itemize}
	\item A – conjunto das arestas do grafo (representam vias e estradas) (A(i) é o i-ésimo elemento)
\begin{itemize}
	\item w – peso da aresta (representa a distância da via)
	\item dest $ \in $  V – vértice de destino da aresta
	\item orig $ \in $  V – vértice de origem da aresta
\end{itemize}
\end{itemize}
\uline{Pontos de interesse} (vértices especiais)
\begin{itemize}
	\item s $ \in $  V – sede/garagem da empresa
	\item pr $ \in $  V – ponto de recarga
	\item lr $ \in $  V – local de recolha de encomenda
	\item le $ \in $  V – local de entrega da encomenda
\end{itemize}

\vspace{\baselineskip}
Ve – conjunto dos veículos pertencentes à frota da empresa (Ve(i) é o i-ésimo elemento)
\begin{itemize}
	\item aut – autonomia, em km

\vspace{\baselineskip}
    \item En – conjunto de encomendas (En(i) é o i-ésimo elemento)
	\item le – local de entrega da encomenda
	\item lr – local de recolha da encomenda
\end{itemize}

%Output
\subsection{Output Data}
Para cada veículo, é retornado um caminho Cam, que representa o melhor caminho (em termos de distância percorrida) para realizar as recolhas e entregas EnV que lhe foram atribuídas.
Para cada veículo
\begin{itemize}
	\item Cam – sequência de nós que representa o melhor caminho para um dado veículo (Cam(i) é o i-ésimo elemento, Cam(i) $ \in $  V)
	\item EnV – conjunto de entregas das quais um veículo ficou encarregue (EnV(i) é o i-ésimo elemento)
\end{itemize}

\end{document}