\chapter{Prospective Solution}

\section{Pre-Processing of Input Data}
Before going through with the problem resolution and trying to find the shortest path for the vehicles, we opt to pre-process the graph, discarding unnecessary nodes and edges. By reducing the number of nodes and edges we analyse we are able improve the temporal spacial efficiency of all other algorithms. 


\subsection{Graph Pre-Processing}
\begin{itemize}
    \item evaluating the connectivity of the graph (algorithm described further ahead), therefore checking for any inaccessible nodes or blocked edges (result of accidents or road works) followed by removal of the irrelevant elements
    \item 
\end{itemize}


\subsection{Order Pre-Processing}
\begin{itemize}
    \item check if any of the pick up or delivery point are not present in the map. If this is the case for any order, it must be removed
\end{itemize}



\section{Problem Identification}

With the graph and the orders pre-processed, there's now a need to find the most efficient way for the vehicles to pick up and deliver the costumers' orders. Before, we have divided the problem into four iterations 
\begin{itemize}
    \item \textbf{I} - Calculation of shortest paths
    \item \textbf{II} -Evaluation of the graph's connectivity
    \item \textbf{III} -Inclusion of recharging points in the paths as needed
    \item \textbf{IV} -Distribution of the orders between the vehicles available to the company
\end{itemize}







