\chapter{Problem Formalization}


\section{Input Data}
The geographical information will be loaded in the form of a map, obtained in OpenStreetMap and converted to a graph.
\paragraph{G = (N, E)} - weighted directed graph that represents the map. It is composed by Nodes and Edges
\begin{itemize}

	\item N - set of Nodes (a node represents an \uline{\textit{interest point}} or simply a point in the road network) (N(i) is the ith element). For each node:
\begin{itemize}
	\item Adj - edges whose origin is N(i)
	\item lat - the latitude of the point it represents in the map
	\item long - the longitude of the point it represents in the map
\end{itemize}

	\item A - set of Edges (an edge represents a way/street/road) (E(i) is the ith element). For each edge:
\begin{itemize}
	\item w - weight (represents the length of the way)
	\item dest - origin node of the edge
	\item orig - destination node of the edge
\end{itemize}

\end{itemize}

\uline{Interest Points} (special nodes)
\begin{itemize}
	\item base - base/garage of the company
	\item rp - recharge points
	\item pup - pick up point
	\item dp - delivery point
\end{itemize}

\paragraph{V} - set of vehicles that make the fleet of the company (Ve(i) is the ith element). For each vehicle:
\begin{itemize}
	\item range - current range of the vehicle
\end{itemize}

\paragraph{Or} - set of orders for the day (Or(i) is the ith element). For each order:
\begin{itemize}
    \item pup - pick up point
	\item dp - delivery point
\end{itemize}


\section{Output Data}
Each vehicle is assigned a \textit{path} that represents the best sequence of nodes for the orders \textit{OrV} he's been assigned to. The path and orders are distributed taking to account the optimization of the distance travelled and the range of each vehicle at any given time.
For each Vehicle V(i):
\begin{itemize}
	\item path – sequence of nodes that represents the path to be taken by a vehicle on a day (path(i) is the ith element, path(i) $ \in $  V)
	\item OrV – set of orders assigned to the vehicle (OrV(i) is the ith element)
\end{itemize}


\section{Restrictions}


\subsection{Input Restrictions}
\paragraph{General}
\begin{itemize}
    \item The limits the sets' indexes are implicitly capped by their sizes\\ \uline{Example:}
    $ 0 \leq i \leq |N| $
    \item 
\end{itemize}

\paragraph{Graph}
\begin{itemize}
    \item Node
    \begin{itemize}
        \item |
        \item $ \forall n \in N, Adj(n) \subseteq E $
        \item $ \forall n \in N, \ang{0} \leq lat \leq \ang{360} $
        \item $ \forall n \in N, \ang{0} \leq long \leq \ang{360} $
    \end{itemize}   
    
    \item{Edge}
    \begin{itemize}
        \item $ \forall e \in E, w \geq 0 $
        \item $ \forall e \in E, orig \in N $
        \item $ \forall e \in E, dest \in N $
    \end{itemize}
\end{itemize}



\subsection{Output Restrictions}


